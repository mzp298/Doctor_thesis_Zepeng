%!TEX root=../Thesis_Zepeng.tex
\chapter{Introduction} \label{chp:1}
\minitoc

\section{General introduction}
The fatigue of metallic structures subjected to cyclic stresses is a phenomenon
which is traditionally studied at two levels. The fatigue is respectively qualified
``low cycle" or ``high cycle" if the load causing the rupture is applied
during a small or a large number of cycles. In turn, ``high cycle fatigue" is divided
in two domains: ``limited endurance" where we speak of the finite lifetime regime
and "unlimited endurance" where the structure can support a number of cycles theoretically
infinite without it breaking. 

The threshold value dividing low- and high-cycle fatigue is somewhat arbitrary, but is generally
based on the raw material’s behavior at the micro-structural level in response to the applied stresses. Low cycle
failures typically involve significant plastic deformation. An example would be reversed 90° bending
of a paper clip. Gross plastic deformation will take place on the first bend, but failure will not occur until
approximately 20 cycles. Plastic deformation does play a role in high cycle fatigue; however, the plastic
deformation is very localized and not necessarily discernible by a macroscopic evaluation of the
component. In summary, while a valve spring designer may consider a failure at 10,000 cycles very short
life, the failure can still be the result of high-cycle fatigue because the material response at the
micro-structural level is the same as in a 10,000,000-cycle failure under lower applied stresses.
Most metals with a body centered cubic crystal structure have a characteristic response to cyclic stresses.
These materials have a threshold stress limit below which fatigue cracks will not initiate. This threshold
stress value is often referred to as the endurance limit. In steels, the life associated with this behavior is
generally accepted to be $2\times10^6$ cycles (\cite{stone2012fatigue}). In other words, if a given stress state does not induce a fatigue failure within the first $2\times10^6$ cycles, future failure of the component is considered unlikely. For spring
applications, a more realistic threshold life value would be $2\times10^7$ cycles (\cite{stone2012fatigue}). Metals with a face center cubic crystal structure (e.g. aluminum, austenitic stainless steels, copper, etc.) do not typically have an endurance
limit. For these materials, fatigue life continues to increase as stress levels decrease; however, a threshold
limit is not typically reached below which infinite life can be expected.

\subsection{Industrial background and motivation}
The search for the best performance at the best cost in mechanics and
transport leads to increasingly
severe conditions of use of the mechanical components. Fatigue failures are widely studied
because it accounts for 90\% of all service
failures due to mechanical causes (\cite{sohar2011lifetime}).  Fatigue failures occur when metal is
subjected to a repetitive or fluctuating
stress and will fail at a stress much lower
than its tensile strength and the process happens without any plastic
deformation (no warning).

These practical problems can lead to the emergence of fatigue for
very high level of stress gradient(for small scale or complex geometric structures) and non-constant multiaxial loading history
. For these so-called "extreme" solicitations, the mechanisms of damage
as well as fatigue resistance levels are mostly unknown.
This poses an important problem during the dimensioning phase since, on the one hand, the
existing endurance criteria struggle to account for behavior for this type of
loading, and on the other hand, the fatigue data that would allow to identify a model
adapted to this problem are almost non-existent.

On the other hand, the mechanical components are generally of complex nature undergoing
complex loads. Manufacturers are looking for a model of lifetime
of their components, which is simple to use, great applicability to metallic materials
and which treats almost all cases of possible loads. In the domain of
limited endurance, very few criteria are proposed. At present, none of them
can be used in design offices, and does fully meet the demand for a tool to
predictive lifetime. Indeed, most of the existing approaches rely on
methods of counting cycles, whose extension to the case of multiaxial stresses turns out to be
difficult or even impossible because of the difficulty of extracting and defining cycles.

Previous work carried out in collaboration with the PSA company indicates that the criteria
of multi-axial endurance used for fatigue dimensioning (Papadopoulos model
and Dang Van's criterion) struggle to account effectively for these very
individuals. (\cite{koutiri2011effet}) It therefore seems essential to characterize the mechanisms
damage of this type of loading and to implement a modeling
able to reflect these particular fatigue conditions.

The aim of this thesis is thus to establish a deterministic model of lifetime
on metal structures working in limited endurance in high cycle fatigue, which handles almost all load cases
(with constant and variable amplitudes) without recourse to cycle counting.

This thesis work is part of a regional project of "Chaire André Citroën"; one of whose objectives is to develop teaching by encouraging initiatives in the automotive sector and confronting students with typical technological innovations and major scientific challenges. The aim of this work is to study the fatigue-related criteria with a large number of cycles, taking into account the effects of variation in time or space. Three contributions were developed:

- Extension of the fatigue criteria to take into account the loading in the vicinity of the working point

- Development and testing of nonlinear accumulation methods of damage.

- Implementation of a strategy for measuring fatigue through a multi-scale analysis of the dissipated energy, thus enabling three-dimensional and complex states of charge to be treated and avoiding the notion of loading cycle.

The study presented in this report focuses more particularly on the last theme, with, as will be seen, a particular emphasis on the effect of micro-structural heterogeneities on fatigue.



The approach for solving the problem posed has four main stages:

$\bullet$ Proposing a strategy to decouple the effects of stress gradient and size effect.

$\bullet$ Review of the existing description of the non-linearity of damage accumulation and history dependent sequence effects.

$\bullet$ Construction of a model of fatigue behavior that accounts for the effects of
microscopic plasticity as well as damage accumulation and history sequencing effects. 

$\bullet$ Numerical simulation with such a model both on cyclic loading conditions and on random loading history.




\subsection{Context and background}
The fatigue of materials with many cycles is one of the phenomena that can lead to
rupture of machine parts or structures in operation. Its progressive character
masked until sudden breakage does not allow easy prediction of the durability of the
structure.

The main factors influencing the fatigue resistance of materials are  numerous
(loading mode, temperature, micro-structural heterogeneities, residual stresses ...),
making it a complex phenomenon to study. A lot of work has been done in the goal of
better understanding the influence of these different factors. One of the main parameters
influential, repeatedly studied, is the damage mechanism of the time varying stress.

Numerous experimental observations made on metallic materials have shown
that the damage mechanisms operating in fatigue with large number of cycles and
leading to breakup are of two categories. In a first step known as the priming step,
micro-plasticity mechanisms, generally operating around heterogeneities specific to the
material (inclusions, porosities, etc.), are the origin of the appearance of micro-cracks. If the load level is high enough, these cracks increase and cross a number of micro-structural barriers (e.g. grain boundaries). When the crack has reached a size
sufficiently large in relation to the microstructure that the plasticized zone, a second phase intervenes where it propagates according to, the laws of the mechanics of the rupture.

Two types of very distinct approaches are often used to model these mechanisms. The
first concerns priming and mainly uses the framework of the mechanics of the
micro-plasticity considered to be the main cause of onset
of a crack. The second uses the fracture mechanics framework to estimate the number
of cycles necessary for the propagation of a pre-existing crack (\cite{koutiri2011effet}). 

In our work we concentrate on the first phase and consider the mechanisms related to the stochastic distribution of pre-existing micro-cracks at different scales which undergo strong plastic yielding in cyclic load history. The number of cycles to failure is determined from the plastic shakedown cycle occurring at these microscales.

\subsection{Outline of the work}
The bibliographical study conducted in the first part of this thesis  (Chapter \ref{chp:2}) aims to
overview the basic multiaxial fatigue criteria and the physical basis of their origin. Models using the elastic adaptation concept, plasticity / damage on the mesoscopic scale as well as energy
are compared. We will show that some loading effects are correctly reflected, however for others, the predictions are very different from one approach to another.

The second part (Chapter \ref{chp:3}) is devoted to the extension of some classic high cycle fatigue (HCF) criteria in order  to take into account a sensitivity of the criteria to stress spatial variations, and second to compare the performances of the extensions through several experimental fatigue tests. The gradient beneficial effect on bending-torsion in comparison with tension-compression is presented. Mechanisms of different approaches are compared and a more practical and simple expression is proposed taking into account the gradient of the stress amplitude and the maximum hydrostatic stress. The generalization of the approach to other multiaxial fatigue criteria is also proposed.  The proposition is then tested and applied to different simple situations such as cantilever rotative bending. The relative errors between the exact solutions and the experimental data are estimated. Biaxial  bending-torsion tests are also simulated to demonstrate the capabilities of the approach. 

The non-linearity of damage accumulation in fatigue is discussed in the third part (Chapter \ref{chp:4}). The objective of this section is to review and use the development life model that takes into account the presence of complex variations of the load cycle. We focus on Chaboche damage accumulation law in case of multiaxial high cycle fatigue. Heuristic formulations with different multiaxial fatigue criteria have been proposed and will be briefly reviewed.

Chapter \ref{chp:5} then considers the problem of handling complex time histories in multiaxial loading. 
Cycle counting method to compare the effect of variable amplitude load histories to fatigue data and curves obtained with simple constant amplitude load cycles is presented, together with different approaches and limitations for handling multiaxial loading.

From this context, we then develop our new model. It is based on a priori given simplified probabilistic description of local material weak points. At each such point, a local plastic model with kinematic hardening is introduced, with a given distribution $p(s)$ of yield weakening factors. To take into account a dependence of the macroscopic fatigue behavior to the hydrostatic stress, the yield limit at each local point is supposed to depend on this macroscopic hydrostatic stress. The model will then suppose that the fatigue accumulation depends on the energy which is dissipated by plasticity of all these points during the loading history.  This energy through all scales $s$ will be combined with the nonlinear damage accumulation laws of chapter \cite{chp:4} to produce a simplified ``multiscale'' model of microscopic damage evolution.

%It will be shown that the use of an probabilistic, taking into account the impurities and hardness in the material method is proposed. A kinematic hardening under the assumptions of associative plasticity is also considered. Usually, the effect of the average stress on loading levels below the yield strength of the material seems well documented in the scientific literature. Beyond this threshold, the mechanisms of damage in play appear less known because less studied. The proposal of our model is to consider a plastic behavior at the mesoscopic scale with a dependence of the yield function not only on the deviatoric part of the stress but also on the hydrostatic part. For fatigue behavior observed under random loading history, the energy approach makes it possible to combine effectively  with the damage accumulation law discussed in Section III.

The sixth chapter of the document deals with the numerical implementation of our method and its validation on different experimental results. Instead of doing the integration directly which can be difficult for complex loading, the Gaussian
Quadrature rule with Legendre points is used to give the value of local dissipated energy rate. Cyclic and random tests on aluminum alloy used for automobile suspension arm is calibrated with our model. Then in the last chapter a multidimensional  application is performed showing the capability of prediction on fatigue life of different material and loading patterns.
