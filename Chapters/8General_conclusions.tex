%!TEX root=../Thesis_Zepeng.tex
\chapter{General conclusions}\label{chp:8}
\minitoc
This study was devoted to the development of a phenomenological and deterministic model of lifetime prediction of structures working in limited endurance under multiaxial stresses of variable amplitude without resorting to the counting of cycles. Special attention is paid to the model so that it can be applied to a wide variety of metallic materials and easy to use for use in design offices.

Our work consisted firstly of describing the methods of calculation of life in limited endurance (finite lifetime regime). We have classified these methods according to the type of stress (uniaxial or multiaxial), the nature of the signal (with constant amplitude or variable amplitude) and whether or not to adopt a cycle count. It was found that the mesoscopic approach, initiated by Dang Van \cite{van1986criterion} and developed later by Papadopoulos \cite{papadopoulos2001long}, gives a physical interpretation of polycyclic fatigue damage. Papadopoulos \cite{papadopoulos2001long}, Morel \cite{FFE:FFE452} and Zarka-Karaouni \cite{de2013comportement} used it and chose cumulated mesoscopic plastic deformation as a variable of damage. The authors assumed that the break occurs when this variable reaches a critical value.

To construct the predictive model of lifetime, we adopted the mesoscopic approach (or macro-meso approach) and used in part the ideas proposed by Papadopoulos and Morel. Indeed, we considered the mesoscopic plasticity induced energy accumulated on the stabilized cycle as variable of the damage. However, unlike the authors, the rupture is not linked to a critical value of the cumulated mesoscopic plastic deformation $\epsilon^{pc}_s$, it is defined by a stochastic distribution of weak points which will undergo strong plastic yielding ,which contribute to energy dissipation and cause
damage, without affecting the overall macroscopic stress. Moreover, the criterion of plasticity at the mesoscopic scale is different. Indeed, the nucleation of microcracks and cracks is a complex phenomenon involving not only plasticity, but also the creation and growth of voids. Although the metallic plasticity is generally independent of the hydrostatic pressure, the growth of the voids depends on the hydrostatic pressure. For this purpose, we have chosen an elastoplastic model with linear kinematic hardening with a mesoscopic elastic limit dependent on the hydrostatic pressure to account for this influence.

A first approach using mesoscopic plastic deformation with a non-zero mean stress and a method of direct calculation of the mesoscopic stabilized cycle is formulated. The difficulty of obtaining an explicit formula for the simple mesoscopic plastic deformation of the stabilized cycle makes the procedure for identifying the parameters of the model complicated. 
%For this, this approach has been abandoned in favor of a second, simpler approach, which does not present this problem. The latter considers a deviatoric mesoscopic plastic deformation and uses the simplified Zarka method on the inelastic behavior of solids for the determination of the mesoscopic stabilized cycle.

In limited endurance, the mesoscopic lifetime criterion was defined for the affine cyclic loads of constant amplitude as a power relation between the stress intensity on the stabilized cycle and the number of cycles at the crack initiation. This criterion was used to identify model parameters using simple loads. An extension of this law to consider the repeated multiaxial loading sequences of variable amplitude is done via a damage factor $D$ depending on energy dissipation and certain parameters characteristic of the material and the loading. Failure is assumed when D = 1.


The dissipated energy to failure per defect  $W_0$ is directly related to the fatigue life scaling. Weakening scales distribution exponent  $\beta$ controls the distribution of weakening scales leading to defining the slope of S-N curve. $\beta$ also takes into account the major damage effect mentioned above.  $\lambda$ is the hydrostatic pressure sensitivity of the elastoplastic material on the mesoscopic scale.  $a$ controls the speed of non-linear damage accumulation. The identification of these parameters involves two steps:

1. Application of the method to the uniaxial case to get 1D best fit (in bending and in torsion): this mainly leads to the identification of $\beta$, which will be used later in an optimization problem.
2. Identification of all parameters of the model by solving a least-square optimization problem (previously obtained relationships): this is to minimize the error between the simulated curve and the experimental Wöhler curve of a test.

The procedure for identifying model parameters requires knowledge of a Wöhler curve (ideally in symmetrical alternate bending) and mean stress effect on fatigue life.

To validate the model, we simulated fatigue tests from the literature and carried out on smooth specimens of four materials (aluminum Al 6082 T6,  steel 30NCD16, steel SM45C and steel 10 HNAP) under multiaxial loadings of constant and variable amplitudes.

A good correlation of the model prediction results with the experimental results was obtained for the proportional loads used, either at constant amplitude or at variable amplitude. The results of prediction are worse for tests carried out on aluminum Al 6082 T6 under non-proportional loads of constant amplitude.

In addition, the model has been applied to study the fatigue strength of AW-6106 T6 aluminum carried out by CETIM(Centre Technique des Industries Mécaniques) under two constant and random signals. The results showed that, in the absence of major damage effect, the life time prediction of the material under random loading is worse. 

The most immediate prospects are validation of the model for different out-of-phase or non-proportional paths. Its application to other industrial structures with a comparison with experimental results is essential for its use in design offices.


\vspace{6pt}
\noindent
\textbf{Acknowledgments}

\vspace{6pt}
We are grateful for the financial and technical support of Chaire PSA.


