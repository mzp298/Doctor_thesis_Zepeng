%!TEX root=../main.tex
\pagestyle{empty}
\newgeometry{top=1cm,bottom=1cm,left=2cm,right=2cm}

\begin{flushleft}
\includegraphics[width=300pt]{figures/SMEMAG.png}

\vspace{20pt}

\begin{mdframed}
\begin{otherlanguage}{french}
\textbf{Titre :} Une nouvelle stratégie pour l'analyse de la fatigue sous chargements multiaxiaux variables

\textbf{Mots clés :} fatigue à grand nombre de cycles; énergie dissipée; approche multi-échelle; plasticité; fatigue à gradient;
cumul non-linéaire de dommage

\textbf{Résumé :} L'objet de ce travail est de proposer une approche multi-échelle de la fatigue fondée sur l'énergie, et
susceptible d'estimer les durées de vie associées à des chargements multidimensionnels variables. Le fondement de la démarche
consiste à supposer que l'énergie dissipée à petite échelle régit le comportement à la fatigue. À chaque point matériel, est
associée une distribution stochastique de points faibles qui sont susceptibles de plastifier et de contribuer à la dissipation
d'énergie sans affecter des contraintes macroscopiques globales. Ceci revient à adopter le paradigme de Dang Van en fatigue
polycyclique. La structure est supposée élastique (ou adaptée) à l'échelle macroscopique. De plus, on adopte à l'échelle
mésoscopique un comportement élastoplastique avec une dépendance de la fonction de charge plastique non seulement
de la partie déviatorique des contraintes, mais aussi de la partie hydrostatique. On considère également un écrouissage
cinématique linéaire sous l'hypothèse d'une plasticité associée. Au lieu d'utiliser le nombre de cycles comme variable
incrémentale, le concept d'évolution temporelle du chargement est adopté pour un suivi précis de l'historique du chargement réel.
L'effet de la contrainte moyenne est pris en compte dans la fonction de charge mésoscopique ; une loi de cumul non linéaire de
dommage est également considérée dans le modèle. La durée de vie à la fatigue est ensuite déterminée à l'aide d'une loi de
phénoménologique fondée sur la dissipation d'énergie mésoscopique issue du cycle d'accommodation plastique.
La première partie du travail a porté sur une proposition d'un modèle de fatigie à gradient de mise en oeuvre plus simple que les
précédents modèles.




\end{otherlanguage}
\end{mdframed}

\vspace{20pt}

\begin{mdframed}
\textbf{Title:} A new strategy for fatigue analysis in presence of general multiaxial time varying loadings

\textbf{Keywords:} High cycle fatigue; dissipated energy; multiscale approach; plasticity ; fatigue gradient;
non-linear damage accumulation

\textbf{Abstract:} The aim of this work is to propose a multi-scale approach to energy-based fatigue, which can estimate lifetimes associated with variable multidimensional loading. The foundation of the approach is to assume that the energy dissipated on a small scale governs the fatigue behavior. Each material point is associated to a stochastic distribution of weak points that are likely to plasticize and contribute to the dissipation of energy without affecting global macroscopic stresses. This amounts to adopting Dang Van's paradigm of high cycle fatigue. The structure is supposed to be elastic (or adapted) on a macroscopic scale. In addition, we adopt on the mesoscopic scale an elastoplastic behavior with a dependence of the plastic load function not only of the deviatoric part of the stresses, but also of the hydrostatic part. Linear kinematic hardening is also considered under the assumption of an associated plasticity. Instead of using the number of cycles as an incremental variable, the concept of temporal evolution of the load is adopted for a precise follow-up of the history of the actual loading. The effect of mean stress is taken into account in the mesoscopic yield function; a law of nonlinear accumulation of damage is also considered in the model. Fatigue life is then determined using a phenomenological law based on mesoscopic energy dissipation from the plastic accommodative cycle. The first part of the work focused on a proposal for a fatigue model with a simpler implementation gradient than the previous models.


\end{mdframed}
\end{flushleft}

\vfill

\begin{minipage}[b]{0.5\textwidth}
\small
\color{color02}
\textbf{Université Paris-Saclay} \\
Espace Technologique / Immeuble Discovery  \\
Route de l'Orme aux Merisiers RD 128 / 91190 Saint-Aubin, France
\end{minipage}
\hfill
\begin{minipage}[b]{0.35\textwidth}
\hfill
\includegraphics[width=35pt]{figures/SMEMAG2.png}
\end{minipage}