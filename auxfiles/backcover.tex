%!TEX root=../main.tex
\pagestyle{empty}
\newgeometry{top=1cm,bottom=1cm,left=2cm,right=2cm}

\begin{flushleft}
\includegraphics[width=300pt]{figures/SMEMAG.png}

\vspace{20pt}

\begin{mdframed}
\begin{otherlanguage}{french}
\textbf{Titre :} Une nouvelle stratégie pour l'analyse de la fatigue en présence de charges multiaxiales générales variant les charges

\textbf{Mots clés :} fatigue; énergie; cycle élevé; plasticité; contrainte moyen

\textbf{Résumé :} L'objet de cet article est de proposer une approche de fatigue basée sur l'énergie qui gère les temps de chargement multidimensionnels variables. Notre pensée fondamentale est de supposer que l'énergie dissipée à petite échelle régit la fatigue à l'échec. La base de notre modèle est de considérer un comportement plastique à l'échelle mésoscopique avec une dépendance de la fonction de rendement non seulement sur la partie déviatorique du stress, mais aussi sur la partie hydrostatique. On considère également un durcissement cinématique sous l'hypothèse d'une plasticité associative. Nous suivons également le paradigme de Dang Van à l'échelle macro. La structure est élastique à l'échelle macroscopique. À chaque point matériel, il existe une distribution stochastique de points faibles qui subiront un fort rendement en plastique, ce qui contribuera à la dissipation d'énergie sans affecter le stress macroscopique global. Au lieu d'utiliser le nombre de cycles, nous utilisons le concept de chargement de l'historique. Pour tenir compte de l'histoire de chargement de la vie réelle plus précisément, l'effet de stress moyen est pris en compte dans la fonction de rendement mésoscopique et la loi d'accumulation de dégats non linéaire est également considérée dans notre modèle. La fatigue sera ensuite déterminée à partir du cycle du shakedown en plastique et d'une loi de fatigue phénoménologique reliant la durée de vie et la dissipation plastique mésoscopique accumulée.

\end{otherlanguage}
\end{mdframed}

\vspace{20pt}

\begin{mdframed}
\textbf{Title:} A new strategy for fatigue analysis in presence of general multiaxial time varying loadings

\textbf{Keywords:} fatigue; energy; high cycle; plasticity; mean stress

\textbf{Abstract:} The object of this paper is to propose an energy based fatigue approach which handles multidimensional time varying loading histories. Our fundamental thought is to assume that the energy dissipated at small scales governs fatigue at failure. The basis of our model is to consider a plastic behavior at the mesoscopic scale with a dependence of the yield function not only on the deviatoric part of the stress but also on the hydrostatic part. A kinematic hardening under the assumption of associative plasticity is also considered. We also follow the Dang Van paradigm at macro scale. The structure is elastic at the macroscopic scale. At each material points, there is a stochastic distribution of weak points, which will undergo strong plastic yielding, which contribute to energy dissipation without affecting the overall macroscopic stress. Instead of using the number of cycles, we use the concept of loading history. To accommodate real life loading history more accurately, mean stress effect is taken into account in mesoscopic yield function and non-linear damage accumulation law are also considered in our model. Fatigue will then be determined from the plastic shakedown cycle and from a phenomenological fatigue law linking lifetime and accumulated mesoscopic plastic dissipation.


\end{mdframed}
\end{flushleft}

\vfill

\begin{minipage}[b]{0.5\textwidth}
\small
\color{color02}
\textbf{Université Paris-Saclay} \\
Espace Technologique / Immeuble Discovery  \\
Route de l'Orme aux Merisiers RD 128 / 91190 Saint-Aubin, France
\end{minipage}
\hfill
\begin{minipage}[b]{0.35\textwidth}
\hfill
\includegraphics[width=35pt]{figures/SMEMAG2.png}
\end{minipage}