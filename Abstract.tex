%&pdflatex
\addcontentsline{toc}{chapter}{Abstract}
% \addtocontents{toc}{\vspace{1em}}  % Add a gap in the Contents, for aesthetics

\begin{center}
  \large Abstract 
\end{center}
The object of this paper is to propose an energy based fatigue approach which handles multidimensional time varying loading histories.

Our fundamental thought is to assume that the energy dissipated at small scales governs fatigue at failure. The basis of our model is to consider a plastic behavior at the mesoscopic scale with a dependence of the yield function not only on the deviatoric part of the stress but also on the hydrostatic part. A kinematic hardening under the assumption of associative plasticity is also considered. We also follow the Dang Van paradigm at macro scale. The structure is elastic at the macroscopic scale. At each material points, there is a stochastic distribution of weak points which will undergo strong plastic yielding, which contribute to energy dissipation without affecting the overall macroscopic stress.

Instead of using the number of cycles, we use the concept of loading history(Chapter~\ref{chp:6}). To accommodate real life loading history more accurately, mean stress effect is taken into account in mesoscopic yield function and non-linear damage accumulation law are also considered in our model. Fatigue will then be determined from the plastic shakedown cycle and from a phenomenological fatigue law linking lifetime and accumulated mesoscopic plastic dissipation.


